\section{Introduction}
\subsection{Contexte}
\subsubsection{Client}

Le client, SLG Classic Cars, est une petite société familiale de restauration et entretien de voitures anciennes. Elle a été initialement fondée en 1976 et a été reprise par les fils en 2015. Ils sont désormais réputé dans leurs domaines et ne cessent de développer leurs activités. La société est basée sur deux sites, un atelier mécanique à Bierwart ainsi qu'un atelier carrosserie à Hingeon. 

\newpara

Dans le cadre de leurs activités, la societé a besoin d'un SI\footnote{\textit{"Le système d’information (SI) est un élément central d’une entreprise ou d’une organisation. Il permet aux différents acteurs de véhiculer des informations et de communiquer grâce à un ensemble de ressources matérielles, humaines et logicielles. Un SI permet de créer, collecter, stocker, traiter, modifier des informations sous divers formats."}\cite{SI}} permettant de gérer de manière efficace les informations relatives aux:
\begin{itemize}
  \item clients 
  \item fournisseurs
  \item véhicules
  \item stock
  \item commandes
  \item devis
  \item factures
  \item fiches de Travail
  \item ...
\end{itemize}

\subsubsection{Solution existante} 

Jusqu'à présent la société utilise une combinaison de deux logiciels:
\begin{description}
  \item[GAD-Garag] : \textit{"Le logiciel garage GAD Garage est un logiciel garage professionnel de gestion commerciale pour la réparation de véhicule , GAD Garage est un logiciel Garage pour Auto , moto et la vente de pièces détachées automobiles (semi-grossiste)."}\cite{GAD}
  \item[SLG-order-manager] : Logiciel propriétaire permettant de générer et réceptionner des commandes. Ce logiciel intéragit directement avec GAD-Garage et permet de simplifier l'encodage des informations relatives au stock. 
\end{description}

\newpara

En raison de divers problèmes liés au logiciel GAD-Garage et un manque de maintenance de celui-ci, la société souhaite développer une nouvelle solution informatique spécialement adaptée à leur façon de travailler et à leurs besoins cités ci-avant. Cette solution doit être conçue de façon à pouvoir être adaptée au fil des années en fonction des nouveaux besoins du client.

\newpage

\subsection{Objectifs}

Au vue de l'envergure du projet et un temps relativement limité quant à la réalisation de ce travail de fin d'études, le client et moi-même avons décidé de définir des objectifs à court et long termes. Les objectifs à court termes sont prioritaires et sont ceux sur lesquels je travaillerai dans le cadre de ce travail de fin d'études. Si le résultat est concluant le projet sera étendu au delà du cadre scolaire et les objectifs à plus long termes seront réalisé. 

\newpara

Bien que les objectifs à long terme soient moins importants, ils ne sont pas indispensables pour autant. Dès lors, durant toute la durée de développement de ce projet, le client continuera d'utiliser ses solutions actuelles. 

\subsubsection{Court terme}

\begin{itemize}
  \item gestion des clients 
  \item gestion des fournisseurs
  \item gestion du stock
  \item gestion de la main d'oeuvre
  \item gestion des commandes 
  \item design "user-friendly"
  \item déploiement
\end{itemize}

\newpara

Ces différents objectifs sont assez vaste et cachent une multitude d'objectifs secondaires tel que la gestion des droits aux ressources, la gestion de conflit d'encodage\footnote{Gérer la possibilité que deux utilisateurs distincte modifier simultanément une même donnée.} et bien d'autres.  

\subsubsection{Long terme}

\begin{itemize}
  \item les autres fonctionnalités (voitures, facturation, ...)
  \item gestion avancée des utilisateurs de la web-App
  \item //TODO
\end{itemize}