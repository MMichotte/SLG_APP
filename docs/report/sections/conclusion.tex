\section{Conclusion}

Ce travail m'a permis d'approfondir et maîtriser plusieurs aspect de la réalisation d'une application pour un client. En particulier:

\vskip 0.5cm

\textbf{1. L'analyse et l'évolutivité} \\Entre l'organisation de réunions avec le client, la traduction de ses besoin en user stories ou encore le design de la base de données, il n'était pas toujours facile de s'adapter aux demandes du client. Néanmoins, j'ai su m'adapter j'ai appris à combiner l'analyse et le développement de façon à être le plus productif possible. 

\vskip 0.5cm
\textbf{2. Le développement} \\Tout au long du développement j'ai appris une grandes quantités de techniques dans le but d'améliorer non seulement la qualité du code mais aussi sa robustesse et lisibilité. 

\vskip 0.5cm
\textbf{3. La mise en production} \\J'avais par le passé déjà utilisé des techniques de déploiement continu mais pas dans le but de réellement déployer l'application dans un milieu de production. Ceci amène une série de challenges qui m'ont obligé à être tres rigoureux. 

\newpara

Bien que l'application ne soit pas terminée, les objects fixés avec le client en début de projet ont été atteints. Les fonctionnalités jusqu'à présent implémentées sont fortement appréciées par le client. Celui-ci m'a d'ailleurs demandé de continuer le travail pour délivrer une application complète. Ceci fera l'objet d'une prochaine collaboration. 

\newpara

Personnellement je suis très content d'avoir pu réaliser ce travail avec un vrai client d'autant que l'application lui sera d'une réelle utilité. Ce travail m'a permis de mettre en oeuvres les notions acquises durant ces années études et d'acquérir une réelle expérience du métier de développeur en tant que freelance. 